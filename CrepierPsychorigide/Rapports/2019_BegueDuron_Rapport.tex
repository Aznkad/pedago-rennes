\documentclass[a4paper,11pt]{article}

\usepackage{fullpage}
\usepackage[T1]{fontenc}
\usepackage[utf8]{inputenc}
\usepackage[francais,french]{babel}

\begin{document}

\title{Rapport d'étonnement : \\le crêpier psychorigide -- École Jules Ferry}
\author{Clara \bsc{Bégué}, Julien \bsc{Duron}}
\date{Mars 2019}

\maketitle

\begin{itemize}
	\item les élèves étaient globalement bienveillants et à l'écoute
	\item attention aux consignes : les élève avaient tendance à mettre le tas de crêpes verticalement et donc se trompaient souvent en mélangeant (crêpes qui tombent ect.)
	\item il faut se mettre au niveau des élèves (à genoux) pour parler
	\item c'est pas parce qu'ils disent qu'ils ont compris que c'est le cas
	\item quand ils y arrivent pas, ils utilisent un algorithme randomisé très peu efficace
	\item certains groupes vont beaucoup plus vite que d'autres : il faut prévoir beaucoup d'extensions (surtout que le crêpier est une activité rapide)
	\item ils peuvent faire des choses très bizarres avec les crêpes quand ils s'ennuient 
	\item penser à faire une fiche scientifique pour les enseignants en plus de la fiche de récap pour les élèves
\end{itemize}

\end{document}