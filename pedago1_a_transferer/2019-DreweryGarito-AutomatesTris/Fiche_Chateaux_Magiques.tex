\documentclass[a4paper,11pt]{article}%

\usepackage{fullpage}%
\usepackage{url}%
\usepackage[T1]{fontenc}%
\usepackage[utf8]{inputenc}%
\usepackage[main=francais]{babel}% % Adjust the main language

\usepackage{graphicx}%

\title{Activité "Châteaux Magiques"}%
\author{Alexandre Drewery \and Yan Garito}%
\date{}%

\begin{document}%

\maketitle%

\section{Résumé de l'activité}

\paragraph{Objectif}
Caractériser un ensemble de mots par un automate fini.

\url{https://fr.wikipedia.org/wiki/Automate_fini}

\paragraph{Niveau}
CM1 - CM2

\paragraph{Durée}
Environ 45 minutes.

\paragraph{Groupes}
Binômes (un trinôme s'il le faut).

\section{Déroulement}

\begin{center}
\begin{tabular}{|l|p{5.5em}|p{15em}|p{6em}|p{6em}|}
  \hline
    \textbf{Durée} 
  & \textbf{Phase} 
  & \textbf{Activité et consignes} 
  & \textbf{Organisation} 
  & \textbf{Matériel} 
  \\
  \hline
  \hline
  <5' 
  & Présentation de l'activité 
  & Contexte : le roi a été enlevé, et les héros du royaume doivent le retrouver dans la salle du trône du château ennemi. Malheureusement, tous les couloirs du château sont protégés par des sortilèges, et les héros ont besoin de l'aide d'un magicien pour se déplacer entre les salles.
  & Oral Collectif 
  & (Aucun)
  \\
  \hline
  5/10' 
  & Consignes 
  & Exemples de déplacements entre salles sur le petit château. Il faut se rendre du pont-levis à la salle du trône. Chaque couloir est associé à une lettre : le magicien doit combiner des lettres pour former un chemin caractérisé par ce mot. 
  & Oral Collectif 
  & Un plan du petit château
  \\
  \hline
  10'/15'
  & Mise en pratique
  & Les élèves se répartissent en binômes : l'un d'entre eux est le magicien qui propose un mot, et l'autre est le héros qui suit le chemin indiqué lettre à lettre (les rôles tournent dans un groupe de trois). On échange les rôles une fois qu'un chemin gagnant est trouvé, pour former, si possible, un autre chemin gagnant. Les encadrants circulent dans la salle pour vérifier que les consignes sont suivies et aiguiller les binômes qui peinent.
  & En groupes de 2 (+ un groupe de 3)
  & Un petit château par groupe
  \\
  \hline
  5'
  & Mise en commun
  & On compare les mots correspondant aux chemins gagnants des différents groupes. Sont-ils tous les mêmes ? Se ressemblent-ils ? Quels sont leurs points communs ? 
  & Oral collectif
  & (Aucun)
  \\
  \hline
  10'
  & Exploration du grand château
  & L'activité reprend dans un château plus grand et plus complexe. Les groupes à l'aise peuvent se voir imposer des contraintes (commencer par une certaine lettre, créer un chemin d'une longueur donnée, ...).
  & En groupe de 2 (+ un groupe de 3)
  & Un grand château par groupe
  \\
  \hline
  5'/10'
  & Mise en commun et conclusion
  & Comparaison des mots formés. Remarques sur la structure du château.
    Section "c'est de l'infomatique parce que...".
  & Oral collectif
  & (Aucun)
  \\
  \hline
\end{tabular}
\end{center}

\end{document}%
