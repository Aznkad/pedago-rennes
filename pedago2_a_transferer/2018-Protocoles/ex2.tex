\documentclass[a4paper,10pt]{article}
\usepackage[utf8]{inputenc}
\usepackage{tikz}
\usepackage{amsmath}
\usepackage{fullpage}

%opening
\title{Exemple 2}
\date{}

\begin{document}

\maketitle

\section{Nombre de joueurs et situation initiale}
\begin{itemize}
	\item Rôles : l'arbitre, un attaquant, et deux joueurs, nommés joueur 1 et joueur 2.
	\item Au début, l'arbitre distribue aléatoirement à chaque joueur et à l'attaquant une clé triangulaire, et une clé carrée correspondante. Ainsi, si le joueur 1 reçoit la clé triangulaire 2, l'arbitre doit aussi lui donner la clé carrée 2, et personne d'autre ne possède cette clé au début.
	\item But : le joueur 1 piochera une carte Secret, et devra l'envoyer au joueur 2 en s'assurant qu'elle reste secrète.
\end{itemize}




\section{Joueur 1}
\begin{enumerate}
	\item Faites un double de votre clé carrée, et envoyez la au joueur 2.
	\item Si vous recevez une boîte contenant une clé carrée, piochez une carte Secret, et enfermez la dans une boîte que vous verrouillerez avec la clé carrée que vous venez de recevoir. Puis envoyez le tout au joueur 2.
\end{enumerate}

\section{Joueur 2}
\begin{enumerate}
	\item Si vous recevez une clé carrée, utilisez la pour fermer une boîte où vous aurez préalablement placé une copie de votre propre clé carrée, puis renvoyez le tout au joueur 1.
	\item Si vous recevez une boîte que vous pouvez ouvrir avec votre clé triangulaire, ouvrez la. Si elle contient une carte Secret, le protocole est terminé.
\end{enumerate}

\section{La faille}
Ce protocole contient une faille :
\begin{itemize}
 \item Lorsque le joueur 1 envoie sa clé carrée au joueur 2, l'attaquant peut bloquer cette clé pour que le joueur 2 ne la reçoive pas, et la garder pour lui.
 \item Par la suite, l'attaquant envoie sa propre clé carrée au joueur 2 (celui-ci ignore que l'attaquant est mal intentionné).
 \item Le joueur 2 envoie alors sa clé carrée dans une boîte que seul l'attaquant pourra ouvrir. Celui-ci peut alors récupérer la clé carrée du joueur 2.
 \item L'attaquant envoie alors sa clé carrée au joueur 1, dans une boîte qu'il ferme avec la clé carrée du joueur 1.
 \item Le joueur 1 envoie alors une carte Secret dans une boîte qu'il ferme avec la clé qu'il a reçue, qui est celle de l'attaquant. L'attaquant peut alors récupérer la boîte et lire le secret. Il peut ensuite l'envoyer au joueur 2 dans une boîte qu'il ferme avec sa clé.
\end{itemize}

\section{En pratique}
Dans cet exemple, le joueur 1 croit qu'il parle avec le joueur 2, alors qu'il parle en réalité avec l'attaquant. Le joueur 2 parle avec quelqu'un, sans avoir besoin de savoir de qui il s'agit.\\
En pratique, le joueur 2 peut être un serveur, et le joueur 1 quelqu'un qui croît parler avec ce serveur ; l'attaquant se fait passer pour le serveur auprès du joueur 1, et peut récupérer tous les secrets que celui-ci veut envoyer au serveur.

\section{Comment corriger l'attaque ?}
Une solution est d'avoir une personne extérieure, qui soit de confiance, et qui puisse permettre de faire le lien entre une clé carrée et l'identité d'une personne, c'est-à-dire qui garantit qu'une personne donnée est la seule à posséder la clé triangulaire qui correspond à une certaine clé carrée.\\Ainsi, au lieu d'envoyer leurs clés carrées, les joueurs demandent à cette personne de confiance de construire une boîte contenant leur clé carrée et leur identité, et verrouillée avec la clé triangulaire de cette personne extérieure (donc signée par cette personne). Tous les joueurs doivent posséder la clé carrée de cette personne extérieure. Il suffit alors de refuser de recevoir des clés triangulaires qui ne sont pas signées par cette personne. Et lorsqu'on en reçoit une signée, elle vient avec une identité et on sait que la clé carrée reçue correspond à la personne dont l'identité est indiquée. Ainsi, dans notre exemple, le joueur 1 saurait, lorsqu'il reçoit une clé, que celle-ci ne vient pas du joueur 2. \\
Une telle personne extérieure est appelée dans la vraie vie une \emph{autorité de certification}.
\end{document}
