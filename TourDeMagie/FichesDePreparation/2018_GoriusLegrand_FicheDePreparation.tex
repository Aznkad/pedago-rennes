% Created 2018-02-15 jeu. 10:56
\documentclass[11pt, landscape]{article}
\usepackage[utf8]{inputenc}
\usepackage[T1]{fontenc}
\usepackage{fixltx2e}
\usepackage{graphicx}
\usepackage{grffile}
\usepackage{longtable}
\usepackage{wrapfig}
\usepackage{rotating}
\usepackage[normalem]{ulem}
\usepackage{amsmath}
\usepackage{textcomp}
\usepackage{amssymb}
\usepackage{capt-of}
\usepackage[margin = .5em]{geometry}
\usepackage{hyperref}
\usepackage{tabularx}
\usepackage{hyperref}
\author{Clément Legrand}
\date{\today}
\title{}
\hypersetup{
 pdfauthor={Clément Legrand},
 pdftitle={},
 pdfkeywords={},
 pdfsubject={},
 pdfcreator={Emacs 24.5.1 (Org mode 8.3.4)}, 
 pdflang={English}}
\begin{document}

\section{Première activité: Arthur cherche un blason}
\url{}
\begin{center}
\begin{tabularx}{\textwidth}{l|l|X|X|l}
Durée & Phases & Activités et consignes & Organisation & Matériel\\
\hline
10' & Explication des consignes & - Présentation des intervenants et des élèves & Oral collectif & - Feuilles de modèles\\
 &  & - Contextualisation de l'activité (Chevaliers) &  & - Blasons vierges\\
 &  & - Distribution du matériel &  & \\
\hline
5' + 5' & Première tentative & - Un maître d'œuvre et un artisan,
dessinent un nouveau blason pour le roi. Le maître d'œuvre dispose de
six modèles et l'artisan de six blasons vierges. Le premier donne des
instructions au second pour dessiner le blason. & En binôme, passage
des intervenants dans les groupes & \\
 &  & - Les élèves échangent de rôles. &  & \\
\hline
5' & Mise en commun & - Les élèves expriment leurs difficultés & Oral collectif & Feuilles d'instructions\\
 &  & - Pistes de reflexion: distibution de la feuille d'instructions &  & \\
 &  & - Lecture des instructions &  & \\
\hline
5' + 5' & Deuxième tentative & - Les élèves réessayent. & En binôme & \\
\hline
10' & Mise en commun et debriefing & - Les élèves expliquent si leurs problèmes se posaient encore & Oral collectif & \\
 &  & - Pourquoi c'est de l'informatique ? &  & \\
 &  &  \hspace{2em} - langage de programmation &  & \\
 &  &  \hspace{2em} - jeu d'instructions &  & \\
 &  &  \hspace{2em} - l'ordinateur n'est pas magique et a une compréhension limitée &  & \\
\end{tabularx}
\end{center}

\newpage

\section{Deuxième activité: Merlin et le donjon}
\url{https://members.loria.fr/MDuflot/files/med/magie.html}
\begin{center}
\begin{tabularx}{\textwidth}{l|l|X|l|l}
Durée & Phases & Activités et consignes & Organisation & Matériel\\
\hline
5' & Présentation de l'activité & - Contextualisation (Merlin
l'Enchanteur) & Oral collectif & Cartes noires d'un coté et blanches de l'autres\\
 &  & - Description du déroulement du tour &  & \\
\hline
5' & Tour de magie devant la classe & - On place les cartes en un grand carré de 5x5 & Oral collectif & \\
 &  & - Le magicien regarde ailleurs pendant que l'apprenti cache un trésor &  & \\
 &  & - L'apprenti complète le carré pour obtenir un carré 6x6 &  & \\
 &  & - Le magicien retrouve le trésor &  & \\
\hline
15' & Réflexion en groupe & - Interrogation des élèves & En demi classe & \\
 &  & - Essai de leurs idées &  & \\
 &  & \hspace{2em} - un trésor trop épais ? &  & \\
 &  & \hspace{2em} - un apprenti complice ? &  & \\
 &  & \hspace{2em} - une mémoire à toute épreuve ? &  & \\
\hline
10' & Appréhension de la solution & - Les élèves peuvent essayer a
leur tour, devenir magicien ou apprenti Le but est de les amener petit
à petit à comprendre que l'ajout des cartes est important.  & En demi classe & \\
\hline
10' & Mise en commun et debriefing & - Confrontation des observations, idées des élèves & Oral collectif & \\
 &  & - Explication du tour &  & \\
 &  & - Pourquoi c'est de l'informatique ? &  & \\
 &  & \hspace{2em} - codes correcteurs d'erreurs, exemples
 d'applications (images, sons, communications\ldots{}) &  & \\
\end{tabularx}
\end{center}
\end{document}