\documentclass{article}
\usepackage[utf8]{inputenc}
\usepackage[T1]{fontenc}
\usepackage[francais]{babel}
\usepackage[left=3cm,right=2cm,top=2cm,bottom=2cm]{geometry}

\usepackage{amsmath}
\usepackage{tikz}

\title{Les blasons - Communiquer avec des langages de programmation}
\author{Joshua Peignier \and Solène Mirliaz}
\date{\today}

\begin{document}
\maketitle
\section{Résumé de l'activité}
Dans cette activité les élèves doivent être capable de transmettre des instructions à un autre élève pour qu'il re-dessine un blason. Les instructions de dessins sont limitées (leur format est précisé dans la fiche d'instructions élève).

\paragraph{Principe} Le but de l'activité est de faire comprendre aux élèves comment, en donnant des instructions simples et claires, on peut faire dessiner à un camarade (qui représente une machine), un dessin qu'il ne voit pas.


\paragraph{Durée} 1h


\paragraph{Concept} Communication - Langage de programmation

\paragraph{Remarque} Les exemples au tableau supposent parfois qu'il y ait deux intervenants, l'un en tant qu'Instructeur, l'autre en tant que Dessinateur. On distingue donc le cas où il n'y en a qu'un de celui où ils sont deux.

\paragraph{Matériel}
\begin{itemize}
  \item Fiche d'exemple (si un seul intervenant/professeur)
  \item Fiche blason exemple (si deux intervenants)
  \item Fiche d'instructions (1/2 page par élève)
  \item Fiche de blasons vierges (1/2 page par élève)
  \item Fiche de blasons modèles (1/4 page par élève)
\end{itemize}


\begin{tabular}{| l | p{2cm} | p{9.5cm} | p{3cm} |}
  %\toprule
  \hline
  Durée & Phase & Activités, consignes & Matériel \\
  \hline
  %\midrule
  5' & Consignes & Professeur: "Aujourd'hui je vous propose d'apprendre à recopier un dessin sans le voir. Mais vous n'allez pas être tout seul. Vous allez travailler par deux. L'un sera l'Instructeur, il vera l'image et donnera les consignes à son camarade, le Dessinateur."\newline
  "Pour éviter de mal se comprendre, l'Instructeur sera limité dans ses instructions. Il peut demander à tracer des lignes, quelques symboles et colorier des zones."& (Aucun) \\ \hline
  \end{tabular}

  \begin{tabular}{| l | p{2cm} | p{9.5cm} | p{3cm} |}
    %\toprule
    \hline
    Durée & Phase & Activités, consignes & Matériel \\
    \hline
  7' & Exemple &
  \begin{itemize}
    \item Si un seul intervenant, celui-ci prend la fiche d'exemple.
          "Cette fiche décrit la liste d'instructions pour dessiner un blason. Je vais la suivre." \newline
          L'intervenant doit commencer par tracer un blason vierge au tableau.
          Il procède ensuite étape par étape selon la fiche.
    \item Si deux intervenant, l'un d'entre eux est désigné "Instructeur".
          Il dispose d'un dessin de blason (grand format) qu'il montre à la classe mais cache au Dessinateur.
          Instructeur : "Je vais maintenant donner au Dessinateur les instructions pour qu'il dessine le blason."
          Il procède étape par étape comme dans le cas un intervenant et le Dessinateur s'exécute.
  \end{itemize}
    Dans les deux cas, on compare à la fin le résultat au tableau et le blason grand format. \newline
    Il faut passer un peu de temps à expliquer le sens des différentes instructions, en particuler celles où on divise une région par un trait.
  & Fiche exemple et fiche blason grand format.\newline Au tableau. \\ \hline
  5' & Distribution du matériel et consignes &
    Professeur: "Vous allez chacun avoir une fiche de consignes qui vous dit exactement ce que vous avez le droit de dire en tant qu'Instructeur."\newline
    "Il y a deux fiches de blasons, l'une vierge, où le Dessinateur va dessiner selon les consignes de l'Instructeur, et une autre avec des blasons parmis lesquels le Dessinateur peut choisir."
  & Fiche d'instructions, fiches de blasons vierges et de blasons modèles. \\\hline
  20' & Activité &
    Les élèves sont chacun à leur tour Instructeur et Dessinateur. À cette étape on autorise les élèves à se poser des questions si les instructions ne sont pas claires. Ils ne doivent par contre pas se montrer les blasons.
  & Fiche d'instructions, fiches de blasons vierges et de blasons modèles. \\ \hline
  5' & Transition & Professeur: "Vous vous êtes rendu compte que vos camarades ne dessinent pas toujours ce que vous souhaitez lorsque vos instructions sont trop vagues. Un ordinateur agira pareil : il faut employer un vocabulaire précis pour qu'il fasse ce que l'on veut."\newline
  "Par exemple, comment faire pour avoir des rayures bien régulières ?"\newline
  (les élèves peuvent proposer leur réponse).
  "Une bonne manière de le faire est de diviser. On part de la plus grosse division et on redivise chaque partie."
  (Montrer l'exemple au tableau).
  "On divise le blason verticalement en deux. Puis, on redivise la partie de gauche verticalement en deux, puis on redivise verticalement chacune de ces deux parties à gauche." (le résultat est donc quatre rayures sur la gauche du blason)& Tableau \\ \hline
  20' & Nouvelles consignes &
  Professeur: "Imaginons maintenant que vous devez donner les instructions à une machine. Elle doit pouvoir l'éxecuter sans avoir à poser de question. Il faut donc tout prévoir, précisemment. Nous allons faire la même chose entre groupes." \newline
  "Vous allez écrire les instructions pour qu'une autre équipe dessine un blasons."
  & Fiche d'instructions, fiches de blasons vierges et de blasons modèles. \\ \hline
  \end{tabular}

  \begin{tabular}{| l | p{2cm} | p{9.5cm} | p{3cm} |}
    %\toprule
    \hline
    Durée & Phase & Activités, consignes & Matériel \\
    \hline
  3' & Mise en commun & Discussion sur les résultats de chacun, réponse aux questions éventuelles & (Aucun) \\ \hline
  5' & Conclusion & Et l'informatique dans tout ça ? \newline
  "Ici, vous avez été à la fois celui qui donne les instructions et celui qui les exécutent. Lorsqu'on s'occupe d'un ordinateur on souhaite que celui-ci fasse exactement ce qu'on lui demande sauf qu'il n'a aucune imagination et ne peut pas le deviner ni poser de question."\newline
  "Les langages de programmation nous permettent de faire faire à l'ordinateur exactement ce que l'on veut. Ceux sont des langages trés précis pour que l'on sache exactement ce que l'ordinateur va faire."\newline
  "C'est aux programmeurs de transformer ce que veut un individu en ce que l'ordinateur va faire." & (Aucun) \\
   \hline

\end{tabular}

\paragraph{Pour aller plus loin} Laisser les élèves inventer leur propres blasons.


\end{document}
