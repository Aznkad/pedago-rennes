\documentclass[a4paper,11pt]{article}%

\usepackage{fullpage}%
\usepackage{url}%
\usepackage[T1]{fontenc}%
\usepackage[utf8]{inputenc}%
\usepackage[main=francais]{babel}% % Adjust the main language

\usepackage{graphicx}%

\title{Activité "Île au trésor"}%
\author{Dylan Bellier \and Jean Jouve}%
\date{}%

\begin{document}%

\maketitle%

\section{Résumé de l'activité}

\paragraph{Objectif}
Utiliser un automate fini pour trouver un ensemble de mots.

\url{https://fr.wikipedia.org/wiki/Automate_fini} (s'il faut plus 
d'informations)

\paragraph{Compétence}
Rechercher dans un graphe, modéliser un problème.

\paragraph{Niveau}
CM1 - CM2

%Un automate est, pour simplifier, un ensemble d'états et de liens entre ces 
%états. Ici, des îles joueront le rôle des états et des courants marins 
%représenterons les liens. Les enfants se déplaceront d'île en île en suivant 
%les courants marins et tenteront de créer une carte des courants entre les 
%îles -- ce qui créera l'automate -- pour atteindre l'île au trésor. Chaque île 
%est représentée par une feuille sur laquelle il est inscrit les îles 
%atteignables depuis celle-ci : il y a au maximum 2 choix, le A et le B. La 
%suite des lettres des chemins choisis depuis la première île forme un mot 
%reconnu par l'automate associé.

\paragraph{Durée}
Environs 45 minutes.

\paragraph{Pour ceux qui ont un peu de mal}
  Si un groupe peine à avancer, on peu leur conseiller de noter leurs décisions 
  sur un bout de papier ou de créer une carte (s'ils ne l'ont pas encore).

\paragraph{Pour ceux qui vont trop vite}
  Si un groupe a trouvé l'île au trésor suffisamment vite, il peut chercher 
  d'autres chemins depuis l'île au pirate jusqu'à l'île au trésor. Il peut 
  éventuellement les chercher tous.

\section{Déroulement}

\begin{center}
\begin{tabular}{|l|p{5.5em}|p{15em}|p{6em}|p{6em}|}
  \hline
    \textbf{Durée} 
  & \textbf{Phases} 
  & \textbf{Activité et consigne} 
  & \textbf{Organisation} 
  & \textbf{Matériel} 
  \\
  \hline
  \hline
  5' 
  & Présentation de l'activité 
  & "Aujourd'hui nous allons faire une chasse au trésor ! Vous êtes tous 
    des pirates et vous cherchez la fameuse île au trésor. Pour la trouver, 
    vous devrez la   chercher d'île en île en suivant des courants marins." 
  & Oral Collectif 
  & (Aucun)
  \\
  \hline
  5/10' 
  & Consignes 
  & \par Présentation d'un problème simple avec trois îles. Nous décrirons les 
    déplacements autorisés par divers exemples. Il y aura l'île au trésor qui 
    est l'objectif, l'île de la tête de mort -- à éviter à tout prix -- et 
    l'île aux pirates qui est le début de l'aventure. Nous demanderons aux 
    enfants de noter les décisions qu'ils prennent au fur et à mesure. Cela 
    créera un itinéraire que l'on pourra suivre pour retrouver ultérieurement 
    l'ile au trésor.
  & Oral Collectif 
  & Un jeu de cartes des îles d'exemple
  \\
  \hline
  10'/15'
  & Mise en pratique
  & \par Les enfants auront un descriptif par île qui leur permettra de savoir 
    où ils peuvent aller depuis cette île. Il y aura sept îles en tout. 
    L'objectif est toujours le même : trouver l'île au trésor en partant 
    de l'île aux pirates mais ils pourraont passer par l'île à la tête de mort.
    \par Nous passerons de groupe en groupe pour vérifier que les enfants 
    respectent les consignes, ne font pas d'erreur, pour les aider s'il ont 
    des difficultées ou pour leur donner plus d'exercice s'ils y arrivent trop 
    bien.
  & En groupe de 4 ou 5
  & Un set de descriptifs par groupe
  \\
  \hline
  15'
  & Ajout d'une carte des îles
  & \par Nous donnerons une carte avec chaque île dessus pour inciter les 
    enfants à cartographier la zone. 
  & En groupe de 4 ou 5
  & Une carte des îles
  \\
  \hline
  5'
  & Conclusion
  & Nous expliqueront en quoi l'activité que nous venont de pratiquer est de 
    l'informatique et présenteront quelques solutions.
  & Oral collectif
  & (Aucun)
  \\
  \hline
\end{tabular}
\end{center}

\end{document}%