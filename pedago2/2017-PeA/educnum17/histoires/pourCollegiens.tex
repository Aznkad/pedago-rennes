\documentclass{article}
\usepackage[francais]{babel} % Le package de Babel pour ecrire en
%                              % francais;
\usepackage[utf8]{inputenc}%
\usepackage[T1]{fontenc}%
\usepackage{url} %
% \usepackage{hyperref} %
% %
% \hypersetup{ %
%   colorlinks=true, %
%   linkcolor=black, %
%   citecolor=black, %
%   urlcolor=black,%
% }%
%

\title{Educnum 2017 - Respect de la vie privée \\
  Histoires pour les 10-14 ans}

\author{Antonin, Louis, Tristan}

\begin{document}

\maketitle 

\textbf{Objectif :} Identifier quels histoires seraient à mettre en
scène pour sensibiliser des 10-14 ans aux risques privacy. 


\section{À qui s'adresse-t-on ? }

\subsection{Références}

Perceptions vie privée des digital natives : 
\begin{itemize}

\item Danah Boyd : Social Privacy in Networked Publics: Teens’
  Attitudes, Practices, and
  Strategies\footnote{\url{https://papers.ssrn.com/sol3/papers.cfm?abstract_id=1925128}}

\item Jean-Marc Manach : Vie privée : le point de vue des ``petits
  cons''
  \footnote{\url{http://www.internetactu.net/2010/01/04/vie-privee-le-point-de-vue-des-petits-cons/}}

\item Jean-Marc Manach : La vie privée, un problème de vieux cons ?
  \footnote{\url{http://www.lemonde.fr/technologies/article/2009/03/17/la-vie-privee-un-probleme-de-vieux-cons_1169203_651865.html}}

  \item Perception privacy des jeunes
    adultes \footnote{\url{http://www.archipel.uqam.ca/8933/}}

\end{itemize}

Outils déjà présents sur EDUCNUM :
\url{https://www.educnum.fr/fr/outils-pedagogiques-vie-privee}


\section{Incubateur à histoires}

\begin{enumerate}
\item Quand c'est l'autorité qui apprend :
  \begin{itemize}
  \item parents
  \item prof de maths
  \item \ldots
  \end{itemize}

\item Quand c'est ``un truc d'adultes'' qui apprend :
  \begin{itemize}
  \item Employeur, assureur, gouvernement, etc
  \item Pertinent ? 
  \end{itemize}

\item Quand c'est un attaquant classique qui apprend : 
  \begin{itemize}
  \item Hacker 
  \end{itemize}

\item Quand c'est le groupe (hors ``amis FB'') qui apprend : 
  \begin{itemize}
  \item Pertinent ? 
  \end{itemize}

\end{enumerate}

\end{document}

%%% Local Variables: 
%%% mode: latex
%%% TeX-master: t
%%% End: 
